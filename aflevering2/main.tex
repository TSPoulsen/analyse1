\documentclass{article}

\usepackage[utf8]{inputenc} % to write æÆøØåÅ.
\usepackage{amsmath,amssymb,amsfonts,mathrsfs,latexsym,mathtools,amsthm}
\include{symbols.tex}
\usepackage[danish]{babel}
\usepackage{todo}
\usepackage{import}
\usepackage[margin=0.9in]{geometry} % Adjust margin
    \setlength\parindent{6pt}
\usepackage{graphicx} % Required for inserting images
\usepackage{enumitem}
    \setenumerate[1]{label={\alph*)}}

% Nice header and footer
\usepackage{fancyhdr}
\usepackage{zref-totpages}

\pagestyle{fancy}
% header rigth 
\fancyhead[l]{Tim Sehested Poulsen}
\fancyhead[r]{tpw705}
%foot center
\fancyfoot[c]{Side \thepage af \ztotpages}

\usepackage[skip=1em]{parskip}

% Custom Commands
\newcommand{\m}[1]{\mathbb{#1}}
\newcommand{\mC}{\m{C}}
\newcommand{\mR}{\m{R}}
\newcommand{\mQ}{\m{Q}}
\newcommand{\mZ}{\m{Z}}
\newcommand{\mN}{\m{N}}

\newcommand{\abs}[1]{\left| #1\right|}
\newcommand{\lr}[1]{\left( #1\right)}
\newcommand{\lrb}[1]{\left[ #1\right]}
\newcommand{\lrc}[1]{\left\{ #1\right\}}
\newcommand{\bigpare}[1]{\bigg( #1\bigg)}
\newcommand{\cbrac}[1]{\left\{ #1\right\}}
\newcommand{\floor}[1]{\left\lfloor #1\right\rfloor}
\newcommand{\ceil}[1]{\left\lceil #1\right\rceil}
\newcommand{\set}[2]{\left\{ #1 \hspace{0.15cm} \textbf{\textbar}  \hspace{0.15cm} #2 \right\} }


% 
\title{Analyse 1 2022/23 - Hjemmeopgave 1}
\author{Tim Sehested Poulsen}

\begin{document}
\textbf{OBS: Alle referencer til sætninger, lemmaer, osv. er fra bogen [CES]}
\section*{Opgave 1}
\subsection*{a)}
Da $f(x) = \sin(\frac{1}{x})$ er en positiv, kontinuert og aftagende 
funktion på intervallet $[1, \infty)$, kan vi bruge sætning 2.20 til at 
konkludere om rækken $\sum_n=1^\infty \sin(\frac{1}{n})$ er konvergent eller divergent.
Det gør vi ved at undersøge det uegentlige integrale $\int_1^\infty \sin(\frac{1}{x}) dx$. 
Ved brug af sætningen for partiel integration\cite[sætning 5.35]{an0} får vi først
\begin{align*}
    \int_1^\infty \sin(\frac{1}{x}) dx &= \lim_{b \to \infty} \int_1^b \sin(\frac{1}{x}) dx \\
    &= \lim_{b \to \infty} \lrb{\sin(\frac{1}{x})x}_1^{\infty} - \int_1^b \frac{-\cos(\frac{1}{x})}{x} dx \\ 
    &= \lim_{b \to 0} \frac{\sin(b)}{b} - \sin(1) - \lim_{b \to \infty}\int_1^b \frac{-\cos(\frac{1}{x})}{x} dx \\ 
\end{align*}
Her kan vi bruge grænseværdien $\lim_{b \to 0} \frac{\sin(b)}{b} = 1$
og sætningen for integration ved substition\cite[sætning 5.39]{an0} til at få
\begin{align*}
    &\lim_{b \to 0} \frac{\sin(b)}{b} - \sin(1) - \lim_{b \to \infty}\int_1^b \frac{-\cos(\frac{1}{x})}{x} dx \\ 
    &= 1 - \sin(1) -  \lim_{b \to \infty} \int_1^{\frac{1}{b}} \frac{\cos(u)}{u} du \\ 
    &= 1 - \sin(1) +  \lim_{b \to \infty} \int_{\frac{1}{b}}^1 \frac{\cos(u)}{u} du \\
\end{align*}
Herfra kan vi se at det originale integrale $\int_1^{\infty} \sin(\frac{1}{x})$ 
er endeligt hvis og kun hvis
$\int_0^1{\frac{\cos(x)}{x}}$ er endeligt. Vi undersøger nu denne grænseværdi
\begin{align*}
&\lim_{b \to \infty} \int_{\frac{1}{b}}^1 \frac{\cos(u)}{u} du \\
&\ge \lim_{b \to \infty} \int_{\frac{1}{b}}^1 \frac{\cos(1)}{u} du \\
&= \lim_{b \to \infty} \cos(1) \lrb{\log(u)}_{\frac{1}{b}}^1 \\
&= \cos(1) \lim_{b \to \infty} \log(1) - \log(\frac{1}{b}) = \infty \\
\end{align*}
Altså kan vi konkludere at $\int_1^{\infty} \sin(\frac{1}{x})$ er divergent, og dermed
er rækken $\sum_{n=1}^{\infty} \sin(\frac{1}{n})$ divergent.


\subsection*{b)}
Vi kan kigge på rækken som en differens af to rækker 
hvor den ene består af $\sum_{n=2}^{\infty} \frac{1}{n^2}$ 
og den anden består af $\sum_{n=2}^{\infty} \frac{1}{n\log(n)}$. 
Fra eksempel 2.23\cite{an1} kan vi genkende den først som en konvergent p-række, 
og med brug af sætning 2.9\cite{an1} kan vi se at rækken vi er interesseret i vil
divergere/konvergere hvis rækken $-\sum_{n=2}^{\infty} - \frac{1}{n\log(n)}$ divergerer/konvergerer.
Vi undersøger derfor denne række ved brug af integraltesten\footnote{
    Sætningen siger specifikt at den kan bruges for rækker som starter i $n=1$,
    men der er intet i beviset der ikke kan generaliseres til en vilkårlig start værdi} 
\cite[sætning 2.20]{an1}, da
$f(x)=\frac{1}{x\log(x)}$ er en positiv, kontinuert og aftagende funktion på intervallet $[2, \infty)$.
Vi får det uegentlige integrale ved brug af integration ved substition \cite[sætning 5.39]{an0} til at være
\begin{align*}
   &\lim_{b \to \infty} \int_{2}^{b} \frac{1}{x\log(x)} dx \\
   &=\lim_{b \to \infty} \int_{\log(2)}^{\log(b)} \frac{1}{x\cdot u}x du \\
   &=\lim_{b \to \infty} \int_{\log(2)}^{\log(b)} \frac{1}{u} du \\
   &=\lim_{b \to \infty} \lrb{\log(u)}_{\log(2)}^{\log(b)} \\
   &=\lim_{b \to \infty} \log\log(b) - \log\log(2)
   =\infty
\end{align*}
Hvorfra vi kan konkludere at dette integrale divergerer, og dermed divergerer rækken 
$\sum_{n=2}^{\infty}\frac{1}{n^2} - \frac{1}{n\log{n}}$.

\subsection*{c)}
Ved at forkorte udtrykket i den absolutte række kan vi se at
\begin{align*}
    &\sum_{n=1}^{\infty} \abs{(-1)^n \frac{(n+c)^2 - (n-c)^2}{n}} \\
    &=\sum_{n=1}^{\infty} \abs{\frac{(n+c)^2 - (n-c)^2}{n}} \\
    &=\sum_{n=1}^{\infty} \abs{\frac{n^2+c^2 + 2nc - n^2 - c^2 + 2nc}{n}} \\
    &=\sum_{n=1}^{\infty} \abs{\frac{4nc}{n}} \\
    &=\sum_{n=1}^{\infty} \abs{4c}
\end{align*}
Hvorfra vi kan se at den eneste tidspunkt denne række vil konvergere er for $c=0$.

\section*{Opgave 2}
Da rækken er alternerende kan vi bruge Leibniz' test \cite[sætning 2.30]{an1} til at se
rækken konvergerer, eftersom den absolutte talfølge er monoton aftagende og går mod 0.
\begin{align*}
&\lim_{n \to \infty}\abs{\frac{(-1)^{n+1}}{n(n+1)}} = \lim_{n \to \infty} \abs{\frac{1}{n(n+1)}} = 0 \\
\end{align*}
At den er monotont aftagende kommer fra at
\begin{align*}
    &\frac{1}{n(n+1)} > \frac{1}{(n+1)(n+2)} \iff n(n+1) < (n+1)(n+2) \\
    &\iff n^2 + n < n^2 + 3n + 2 \\
    &\iff n < 3n + 2
\end{align*}
Altså vil rækken konvergere og ved brug af observation 2.31\cite{an1} kan vi se at
den vil konvergere mod $s \in [s_4, s_5]$, hvor $s_4$ og $s_5$ er de delsummer vi får
af de først $4$ og $5$ led i rækken. De er givet ved
\begin{align*}
    s_4 &= \frac{1}{1\cdot2} - \frac{1}{2\cdot3} + \frac{1}{3\cdot4} - \frac{1}{4\cdot5} = \frac{11}{30} \\
    s_5 &= \frac{1}{1\cdot2} - \frac{1}{2\cdot3} + \frac{1}{3\cdot4} - \frac{1}{4\cdot5} + \frac{1}{5\cdot6} = \frac{4}{10}
\end{align*}
Altså konvergerer den mod $s \in [\frac{3}{10}, \frac{4}{10}]$.

\section*{Opgave 3}
\subsection*{a)}
Ved brug af rodtesten \cite[sætning 2.26]{an1} kan vi undersøge for hvilke $a,b > 0$ at rækken vil konvergere.
Vi undersøger derfor følgende grænseværdi\footnote{Der bliver brugt adskillige regneregler for grænseværdier, alle er fundet fra \cite[kapitel 2.4]{an0}}
\begin{align*}
    &\lim_{n \to \infty} \sqrt[n]{\frac{n^{an+b}}{(an+b)^n}}
    =\lim_{n \to \infty} \frac{n^{a+\frac{b}{n}}}{an+b} \\
    &=\lim_{n \to \infty} \frac{n^{a-1}\cdot n^{\frac{b}{n}}}{a+\frac{b}{n}}
    =\frac{\lim_{n \to \infty} n^{a-1}\cdot n^{\frac{b}{n}}}{\lim_{n \to \infty} a+\frac{b}{n}} \\
    &=\frac{\lim_{n \to \infty} n^{a-1}\cdot n^{\frac{b}{n}}}{\lim_{n \to \infty} a+\frac{b}{n}}
    =\frac{\lim_{n \to \infty} n^{a-1}\cdot n^{\frac{b}{n}}}{\lim_{n \to \infty} a+\frac{b}{n}} \\
    &=\frac{\lim_{n \to \infty} n^{a-1}\cdot \lim_{n \to \infty}n^{\frac{b}{n}}}{a}
    =\frac{\lim_{n \to \infty} n^{a-1}}{a}
\end{align*}
Herfra kan vi konkludere at når $a < 1$ vil rækken konvergere mod $0$ da $n^{a-1} \to 0$.
Derudover når $a > 1$ vil rækken divergere mod $\infty$. Altså kan vi konkludere at den original række
vil konvergere for $a < 1$ og et vilkårligt $b$.
\subsection*{b)}
Vi undersøger så tilfældet for $a = 1$, hvor rodtesten er inkonklusiv. Vi kan omskrive udtrykket og se at

\begin{align*}
\frac{n^{n+b}}{(n+b)^n} 
=\lr{\frac{n \cdot n^{\frac{b}{n}}}{n+b}}^n
=\lr{\frac{n^{\frac{b}{n}}}{1+\frac{b}{n}}}^n
=\frac{n^b}{\lr{1+\frac{b}{n}}^n}
\end{align*}
Undersøger man $\lim_{n \to \infty} (1+ \frac{b}{n})^n$ vil man kunne få stærk inspiration fra eksempel 1.47 \cite{an1}
og se at 
\begin{align*}
    &\log\lr{(1+ \frac{b}{n})^n}
    = n \cdot \log(1+ \frac{b}{n}) 
    = \frac{\log(1+ \frac{b}{n})}{\frac{1}{n}} \\
    &= \frac{\log\lr{\frac{\frac{1}{b}}{\frac{1}{b}} + \frac{\frac{1}{n}}{\frac{1}{b}}}}{\frac{1}{n}}
    = \frac{\log\lr{\frac{\frac{1}{b} + \frac{1}{n}}{\frac{1}{b}}}}{\frac{1}{n}}
    = \frac{\log\lr{\frac{1}{b} + \frac{1}{n}} - \log \lr{\frac{1}{b}}}{\frac{1}{n}}
\end{align*}
Hvilket er en differenskvotient, som vi ved er $\log'(\frac{1}{b}) = b$ når $n \to \infty$.
Altså kan vi konkludere at 
\begin{align*}
&\lim_{n \to \infty} (1+ \frac{b}{n})^n
=\lim_{n \to \infty} \exp \lr{\log \lr{(1+ \frac{b}{n})^n}}  \\
&=\exp \lr{\lim_{n \to \infty} \log \lr{(1+ \frac{b}{n})^n}} 
=\exp \lr{b}  \\
\end{align*}
Vi trækker her på kontinuitet af $\exp$ og $\log$ samt sætning 1.45 \cite{an1}. 
Vi kan hurtigt konkludere at både $exp$ og $\log$ er kontinuerte funktioner i følgens elementer
eftersom $(1+\frac{b}{n})^n > 1$ for $b>0$ hvor begge funktioner er kontinuerte.

Så til konklusion må vi have at følgen $\lrc{\frac{n^{n+b}}{(n+b)^n}}_{n \in \mN}$ er divergent uanset valget af $b> 0$,
og da det er et nødvendigt kriteriere for en talrække at konvergere at følgen skal konvergere mod $0$\cite[sætning 2.2]{an1} må
det gælde at talrækken er divergent.


\section*{Opgave 4}

%bibliography
\bibliographystyle{plain}
\bibliography{refs.bib}

\end{document}