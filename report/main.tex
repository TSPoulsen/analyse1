\documentclass{article}

\usepackage[utf8]{inputenc} % to write æÆøØåÅ.
\usepackage{amsmath,amssymb,amsfonts,mathrsfs,latexsym,mathtools,amsthm}
\include{symbols.tex}
\usepackage[danish]{babel}
\usepackage{todo}
\usepackage{import}
\usepackage[margin=0.9in]{geometry} % Adjust margin
    \setlength\parindent{6pt}
\usepackage{graphicx} % Required for inserting images
\usepackage{enumitem}
    \setenumerate[1]{label={\alph*)}}

% Custom Commands
\newcommand{\m}[1]{\mathbb{#1}}
\newcommand{\mC}{\m{C}}
\newcommand{\mR}{\m{R}}
\newcommand{\mQ}{\m{Q}}
\newcommand{\mZ}{\m{Z}}
\newcommand{\mN}{\m{N}}

\newcommand{\abs}[1]{\left| #1\right|}
\newcommand{\lr}[1]{\left( #1\right)}
\newcommand{\bigpare}[1]{\bigg( #1\bigg)}
\newcommand{\cbrac}[1]{\left\{ #1\right\}}
\newcommand{\floor}[1]{\left\lfloor #1\right\rfloor}
\newcommand{\ceil}[1]{\left\lceil #1\right\rceil}
\newcommand{\set}[2]{\left\{ #1 \hspace{0.15cm} \textbf{\textbar}  \hspace{0.15cm} #2 \right\} }


% 
\title{Analyse 1 2022/23 - Hjemmeopgave 1}
\author{Tim Sehested Poulsen}

\begin{document}
\section*{Opgave 1}
\subsection*{a)}
Polarformen for et komplekst tal er givet ved $z = |z|e^{i\theta}$, hvor $\theta$ er argumentet for $z$.
Så i vores tilfælde svarer det til at $|z| = \sqrt{(-1)^2 + \sqrt{3}^2} = \sqrt{4}= 2$ og argumentet er det $\theta$ som løser
\begin{align*}
\sin(\theta) &= \frac{\sqrt{3}}{|z|} = \frac{\sqrt{3}}{2} \\
\cos(\theta) &= \frac{1}{|z|} = \frac{1}{2} \\
\end{align*}
Det kan udregnes til at $\theta = \frac{\pi}{3}$. Altså er polarformen for $z= 2e^{i\frac{\pi}{3}}$.
Vi kan nu udregne $z^2 = 2^2e^{i\frac{2\pi}{3}} = 4e^{i\frac{2\pi}{3}}$, og ligeledes
$z^3 = 2^3e^{i\frac{3\pi}{3}} = 8 e^{i\pi} = -8$ samt
$z^6 = 2^6e^{i\frac{6\pi}{3}} = 64 e^{2\pi} = 64$.
\subsection*{b)}
Vi kan se at $z^n$ vil være et reelt tal når argumentet er en multiplicetet af $pi$, og da argumentet for 
$z^n$ er givet ved $n\frac{\pi}{3}$ vil det ske når $\frac{n}{3}$ er et heltal, altså for $n=0,3,6,9,\dots$. Altså 3 tabellen.

\subsection*{c)}
\Todo{Skriv denne pænt}
Eftersom $|(c\cdot z)^n| = |c\cdot z|^n$ vil den absolutte værdi stige hvis $|c \cdot z| > 1$, altså hvis $|c| > \frac{1}{2}$, 
da vi ved at $|z| = 2$. Altså er følgen ubegrænset for $|c| > \frac{1}{2}$ og da den logiske kontraponering af lemma 1.37 siger at
ubegrænsede følger er divergente, kan vi så konkludere at for $c \in \mC$ med $|c| > \frac{1}{2}$ er følgen divergent.
For $|c| =\frac{1}{2}$ kan vi se at $|(c \cdot z)^n| = (\frac{1}{2} \cdot 2)^n = 1^n = 1$.

Hvis vi så kigger på et $c = \frac{1}{2} e^{i\phi}$, hvor $\phi \in [ 0,2\pi )$ kan vi se at $(c \cdot z)^n = e^{n\cdot (\phi + \frac{\pi}{3})}$, hvor proposition 1.30 siger at
hvis er $ \frac{\phi + \frac{\pi}{3}}{2\pi}$ er et rationelt tal, hvor den kan skrives som uforkortelig brøk
$\frac{x}{2\pi} = \frac{p}{q}$ vil have $q$ fortætnigspunkter og hvis argumentet er irrationelt så vil den have uendeligt mange fortætnigspunkter.
Da vi ved fra lemma 1.35 at en konvergent følge vil have et entydigt fortætnigspunkt så kan følgen ikke være konverkent hvis der er
adskillige fortætnigspunkter. Så i dette tilfælde kan vi sige at følgen kun vil have et fortætnigspunkt hvis
$\frac{\phi + \frac{\pi}{3}}{2\pi} = \frac{\phi}{2\pi} + \frac{1}{6}$ er et helttal. For et $\phi \in [0, 2\pi)$ kan det kun ske når 
$\phi = \frac{5}{3}\pi$. Altså har jeg at hvis $|c| = \frac{1}{2}$ vil følgen kun have et fortætningspunkt hvis $c = \frac{1}{2}e^{i\frac{5\pi}{3}}$.
Det kan nu let ses at $c \cdot z = \frac{1}{2}e^{i\frac{5}{3}\pi} \cdot 2 e^{i \frac{1}{3}\pi} = e^{i \frac{6}{3}\pi} = e^{i2\pi} = 1$.
Altså vil $(c \cdot z)^n = 1^n = 1$ definere en konvergent følge med grænseværdi $1$.

For $c \in \mC$ med $|c| < \frac{1}{2}$ kan man se at
$|c\cdot z| < 1$ og da vi ved at for $x \in \mR$ med $|x| < 1$ har vi at $x^n$ konvergerer mod $0$, altså vil
vil $|c \cdot z|$ konvergere mod $0$ og derfor vil $c \cdot z$ også konvergere mod $0$.

I konklusion vil det gælde for et hvert $ c \in K_1 := \set{w \in \mC}{|w| < \frac{1}{2} \lor w = \frac{1}{2}e^{i\frac{5}{3}\pi}}$

\subsection*{c)}
Da lemma 1.65 siger at enhver delfølge af en konvergent følge vil også være konvergent,
kan vi bruge vores resultat fra delopgave c) til at sige at for alle $c \in K_1$
vil alle delfølger af $\{ a_n \}_{n \in \mN}$ konvergere. Jævnfør lemma 1.30 kan 
vi også sige at for et hvert $c \in \mC$ hvor $|c| = \frac{1}{2}$ vil der også være mindst
et fortætnigspunkt, og lemma 1.63 siger at for hvert fortætnigspunkt for en følge
vil der være en delfølge der konvergerer mod det fortætnigspunkt. Altså vil 
der for et hvert $c \in K_2 := \set{w \in \mC}{|w| \le \frac{1}{2}}$ være en delfølge som konvergerer. 
Derudover kan vi også konkludere at for $|c| > \frac{1}{2}$ vil følgen være ubegrænset og derfor divergent, eftersom
$|c \cdot z| = |c| \cdot |z| > \frac{1}{2} \cdot 2 = 1$, altså vil normen af følgens elementer vokse og den er derfor ubegrænset.
Eftersom der generelt vil gælde at $|a_n| < |a_{n+i}|$ for $i \in \mN$ vil enhver delfølge også være voksende og ubegrænset.
Så vi kan konkludere at hvis $|c| > \frac{1}{2}$ vil der ikke være nogen delfølge der konvergerer, og $K_2$ vil være mængden af alle komplekse tal
hvorom der findes en delfølge af $\{ a_n \}_{n \in \mN}$ der konvergerer.


\section*{Opgave 2}
\subsection*{a)}
Hvis vi ser på $a_n = \frac{1}{2}(n + \frac{8}{n})$ som funktionen $f: (0, \infty) \to \mR$ givet ved $f(x) = \frac{1}{2}(x + \frac{8}{x})$, 
kan vi hurtigt se af denne funktion er differentiabel og at $f'(x) = \frac{1}{2}(1 - \frac{8}{x^2})$. 
Vi kan så lave en ekstremums undersøgelse af $f$ for at finde hvor den er mindst, hvis den er nedad begrænset.
Jeg løser $f'(x) = 0$ og får
\begin{align*}
    \frac{1}{2}(1 - \frac{8}{x^2}) = 0 \iff 1 - \frac{8}{x^2} = 0 \\
    \iff 1 = \frac{8}{x^2} \iff x^2 = 8 \iff x = \sqrt{8}
\end{align*}
og da $2 = \sqrt{4} < \sqrt{8} < \sqrt{9} = 3$, kan vi finde hældningen på begge sider af ekstremumspunktet $x = \sqrt{8}$.
For $x = 2$ vil $f'(x) = -\frac{3}{2} < 0$ og for $x = 3$ vil $f'(x) = \frac{1}{18} > 0$, altså antager $f$ et globals minimum i $x = \sqrt{8}$.
Da vi ikke kan evaluere $a_n$ i $\sqrt{8}$ må den mindste værdi være for enten $n = 2$ eller $n = 3$, og da $a_2 = 4$ og $a_3 = \frac{11}{3}$,
kan vi konkludere at $a_n$ er mindst for $n=2$.


\subsection*{b)}
Det kan ses at følgen $\{a_n - \floor{a_n} \}_{n \in \mN}$ vil bestå af elementer i intervallet $[0, 1)$ og kigger man på de lige indekser (større end 4),
altså delfølgen af $n_k = 2(k+2)$ for $k \in \mN$ vil man se at elementerne i delfølgen er givet ved 
\[
    a_{n_k} - \floor{a_{n_k}} = \frac{1}{2} (2(k+2) + \frac{8}{2(k+2)}) - \floor{\frac{1}{2} (2(k+2) + \frac{8}{2(k+2)})}= \frac{2}{k+2} - \floor{\frac{2}{k+2}} = \frac{2}{k+2} 
\]
Vi kan bruge sætning 1.39 til at se at delfølgen her vil konvergere mod grænseværdien for produktet af følgerne $\{2\}_{k \in \mN}$ og $\{ \frac{1}{k+2} \}_{k \in \mN}$
hvilket har $2$ og $0$ som grænseværdier, så delfølgen konvergerer mod $0$.

\subsection*{c)}
Hvis jeg kigger på delfølgen af alle de ulige indekser (større end 8), vil jeg se at $n_k = 2(k+3)+1 $ for $k \in \mN$, og at
\begin{align*}
    a_{n_k} - \floor{a_{n_k}} &= \frac{1}{2} (2(k+3) + 1 + \frac{8}{2(k+3)+1}) - \floor{\frac{1}{2} (2(k+3) + 1 + \frac{8}{2(k+3)+1})}  \\
    &= \frac{1}{2} + \frac{4}{2(k+3)+1} - \floor{\frac{1}{2} + \frac{4}{2(k+3)+1}} = \frac{1}{2} + \frac{4}{2k+7} 
\end{align*}
hvilket kan ses eftersom $ 0 < \frac{4}{2(k+3)+1} < \frac{1}{2}$.
Igen kan vi bruge sætning 1.39 til at genkende dette som summen af to konvergente følger, nemlig $\{ \frac{1}{2} \}_{k \in \mN}$ og 
$\{ \frac{4}{2k+7} \}_{k \in \mN}$ som hver især har grænseværdierne $\frac{1}{2}$ og $0$, derfor vil denne delfølge konvergere mod $\frac{1}{2}$.
Vi kan nu bruge lemma 1.63 til at konkludere at der må være minimum 2 fortætnigspunkter for $\{ a_n - \floor{a_n} \}_{n \in \mN}$ og derfor kan vi sige at
følgen er divergent da lemma 1.35 siger at en konvergent følge vil have et entydigt fortætningspunkt.


\section*{Opgave 3}
\subsection*{a)}

\end{document}