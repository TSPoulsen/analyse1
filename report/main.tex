\documentclass{article}

\usepackage[utf8]{inputenc} % to write æÆøØåÅ.
\usepackage{amsmath,amssymb,amsfonts,mathrsfs,latexsym,mathtools,amsthm}
\include{symbols.tex}
\usepackage[danish]{babel}
\usepackage{todo}
\usepackage{import}
\usepackage[margin=0.9in]{geometry} % Adjust margin
    \setlength\parindent{6pt}
\usepackage{graphicx} % Required for inserting images
\usepackage{enumitem}
    \setenumerate[1]{label={\alph*)}}

% Custom Commands
\newcommand{\m}[1]{\mathbb{#1}}
\newcommand{\mC}{\m{C}}
\newcommand{\mR}{\m{R}}
\newcommand{\mQ}{\m{Q}}
\newcommand{\mZ}{\m{Z}}
\newcommand{\mN}{\m{N}}

\newcommand{\abs}[1]{\left| #1\right|}
\newcommand{\pare}[1]{\left( #1\right)}
\newcommand{\bigpare}[1]{\bigg( #1\bigg)}
\newcommand{\cbrac}[1]{\left\{ #1\right\}}
\newcommand{\floor}[1]{\left\lfloor #1\right\rfloor}
\newcommand{\ceil}[1]{\left\lceil #1\right\rceil}


% 
\title{Analyse 1 2022/23 - Hjemmeopgave 1}
\author{Tim Sehested Poulsen}

\begin{document}
\section*{Opgave 1}
\subsection*{a)}
Polarformen for et komplekst tal er givet ved $z = |z|e^{i\theta}$, hvor $\theta$ er argumentet for $z$.
Så i vores tilfælde svarer det til at $|z| = \sqrt{(-1)^2 + \sqrt{3}^2} = \sqrt{4}= 2$ og argumentet er det $\theta$ som løser
\begin{align*}
\sin(\theta) &= \frac{\sqrt{3}}{|z|} = \frac{\sqrt{3}}{2} \\
\cos(\theta) &= \frac{1}{|z|} = \frac{1}{2} \\
\end{align*}
Det kan udregnes til at $\theta = \frac{\pi}{3}$. Altså er polarformen for $z= 2e^{i\frac{\pi}{3}}$.
Vi kan nu udregne $z^2 = 2^2e^{i\frac{2\pi}{3}} = 4e^{i\frac{2\pi}{3}}$, og ligeledes
$z^3 = 2^3e^{i\frac{3\pi}{3}} = 8 e^{i\pi} = -8$ samt
$z^6 = 2^6e^{i\frac{6\pi}{3}} = 64 e^{2\pi} = 64$.
\subsection*{b)}
Vi kan se at $z^n$ vil være et reelt tal når argumentet er en multiplicetet af $pi$, og da argumentet for 
$z^n$ er givet ved $n\frac{\pi}{3}$ vil det ske når $\frac{n}{3}$ er et heltal, altså for $n=0,3,6,9,\dots$. Altså 3 tabellen.

\subsection*{c)}
\Todo{Skriv denne pænt}
Eftersom $|(c\cdot z)^n| = |c\cdot z|^n$ vil den absolutte værdi stige hvis $|c \cdot z| > 1$, altså hvis $|c| > \frac{1}{2}$, 
da vi ved at $|z| = 2$. Altså er følgen ubegrænset for $|c| > \frac{1}{2}$ og da den logiske kontraponering af lemma 1.37 siger at
ubegrænsede følger er divergente, kan vi så konkludere at for $c \in \mC$ med $|c| > \frac{1}{2}$ er følgen divergent.
For $|c| =\frac{1}{2}$ kan vi se at $|(c \cdot z)^n| = (\frac{1}{2} \cdot 2)^n = 1^n = 1$.

Hvis vi så kigger på et $c = \frac{1}{2} e^{i\phi}$, hvor $\phi \in [ 0,2\pi )$ kan vi se at $(c \cdot z)^n = e^{n\cdot (\phi + \frac{\pi}{3})}$, hvor proposition 1.30 siger at
hvis er $ \frac{\phi + \frac{\pi}{3}}{2\pi}$ er et rationelt tal, hvor den kan skrives som uforkortelig brøk
$\frac{x}{2\pi} = \frac{p}{q}$ vil have $q$ fortætnigspunkter og hvis argumentet er irrationelt så vil den have uendeligt mange fortætnigspunkter.
Da vi ved fra lemma 1.35 at en konvergent følge vil have et entydigt fortætnigspunkt så kan følgen ikke være konverkent hvis der er
adskillige fortætnigspunkter. Så i dette tilfælde kan vi sige at følgen kun vil have et fortætnigspunkt hvis
$\frac{\phi + \frac{\pi}{3}}{2\pi} = \frac{\phi}{2\pi} + \frac{1}{6}$ er et helttal. For et $\phi \in [0, 2\pi)$ kan det kun ske når 
$\phi = \frac{5}{3}\pi$. Altså har jeg at hvis $|c| = \frac{1}{2}$ vil følgen kun have et fortætningspunkt hvis $c = \frac{1}{2}e^{i\frac{5\pi}{3}}$.
Det kan nu let ses at $c \cdot z = \frac{1}{2}e^{i\frac{5}{3}\pi} \cdot 2 e^{i \frac{1}{3}\pi} = e^{i \frac{6}{3}\pi} = e^{i2\pi} = 1$.
Altså vil $(c \cdot z)^n = 1^n = 1$ definere en konvergent følge med grænseværdi $1$.

For $c \in \mC$ med $|c| < \frac{1}{2}$ kan man se at
$|c\cdot z| < 1$ og da vi ved at for $x \in \mR$ med $|x| < 1$ har vi at $x^n$ konvergerer mod $0$, altså vil
vil $|c \cdot z|$ konvergere mod $0$ og derfor vil $c \cdot z$ også konvergere mod $0$.

\section*{Opgave 2}
\subsection*{a)}


\section*{Opgave 3}
\subsection*{a)}
\maketitle

\end{document}